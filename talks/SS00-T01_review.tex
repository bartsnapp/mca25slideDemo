%%%%%%%%%%%%%%%%%%%%%%%%% Begin Preamble %%%%%%%%%%%%%%%%%%%%%%%%%
\documentclass[11pt]{ximera}
%%%%%%%%%%%%%%%%%%%%%%%%% Begin Topmatter %%%%%%%%%%%%%%%%%%%%%%%%
\title{SS00-T01 \\ The Mathematics of the Late Pleistocene}
\author{Oog von Cave Enjoyer III}
\affil{Swedish Spelunking Academy}
\date{}
\begin{document}
\begin{abstract}
\end{abstract}
\pagenumbering{gobble}
\maketitle
%%%%%%%%%%%%%%%%%%%%%%%%% Begin Abstract %%%%%%%%%%%%%%%%%%%%%%%%%
Recent explorations of cave networks in the fictional realm of
Narnia~\cite{OvCE} have indirectly revealed that the Neanderthal
inhabitants refused to accept the existence of any integer greater than
three.  Surprisingly, without a working framework of the 
set~\(\mathbb Z\) of integers they came up with the following formula
for the surface area of a torus with minor radius \(r\) and major 
radius~\(R\):
    \[\mathscr A = \int_{0}^{r}\frac{8 \pi Rr}{\sqrt{r^2-y^2}}\,dy.\]
This seemed quite important to the cave-dwelling inhabitants as
donuts~\cite{ALPW} were a major food source and applying a sweetening 
agent to the surface was a critical step in the process of making them 
palatable.  This provides us with another example of the need for
survival driving intelligent organisms to the acceptance of concepts
outside of their standard canon.
\begin{thebibliography}{2}
    \bibitem{OvCE} Oog von Cave Enjoyer III, {\it Caves of Narnia,\/} 
    Journal of Paleolithic Nonsense, 42 (2022), pp. 33--66, 
    \url{https://doi.org/12.3456/ABC00000.2022.0000000}.
    \bibitem{ALPW} A. \L ofty Phictional Righter, {\it Donuts: Miracle
    Food or Empty Calories?,\/} Nutrition Quarterly, 28 (1995), pp. 
    465-637.
\end{thebibliography}
%%%%%%%%%%%%%%%%%%%%%%%%% End Abstract %%%%%%%%%%%%%%%%%%%%%%%%%%%
\end{document}
